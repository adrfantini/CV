              %*****************************************%
              %                                         %
              %       Modello di curriculum vitae       %
              %          di Lorenzo Pantieri ©          %
              %                                         %
              %       versione: 26 settembre 2012       %
              %                                         %
              %*****************************************%


% I seguenti commenti speciali impostano:
% 1. utf8 come codifica di input,
% 2. PDFLaTeX come motore di composizione;
% 3. il controllo ortografico italiano per l'editor.

% !TEX encoding = UTF-8 Unicode
% !TEX TS-program = pdflatex
% !TEX spellcheck = it-IT

\documentclass[11pt,%                        % corpo del font: ci sono anche '10pt' e '12pt'
               a4paper,%                     % carta A4
               sans,%                        % famiglia di font: c'è anche 'roman'
               ]{moderncv}                   % buona classe per CV

\usepackage[T1]{fontenc}                     % codifica dei font:
                                             % richiede una distribuzione completa di LaTeX,

\usepackage[utf8]{inputenc}                  % codifica di input:
                                             % va accordata con le preferenze dell'editor

\usepackage[italian]{babel}                  % per scrivere in italiano

\usepackage{microtype}                       % microtipografia

% \usepackage{lipsum}                          % genera testo fittizio

\moderncvstyle{casual}                      % tema di moderncv:
                                             % oltre a 'classic', ci sono 'casual' (predefinito), 'oldstyle' e 'banking'

\moderncvcolor{orange}                         % colori di moderncv:
                                             % oltre a 'blue' (predefinito), ci sono 'orange', 'green', 'red', 'purple',
                                             % 'grey' e 'black'

%\nopagenumbers{}                            % decommenta per disabilitare la numerazione automatica delle pagine
                                             % per CV più lunghi di una pagina

\usepackage[scale=0.75]{geometry}            % imposta i margini

\setlength{\hintscolumnwidth}{3cm}          % regola la larghezza della colonna con le date

%\setlength{\makecvtitlenamewidth}{10cm}     % nello stile 'classic', regola la larghezza riservata al nome;
                                             % tale lunghezza va modificata con cautela per evitare
                                             % sovrapposizioni con i dati personali

\firstname{Adriano}                          % nome

\familyname{Fantini}                        % cognome

\title{Curriculum vitae}                     % titolo del CV (opzionale: rimuovi la riga se non lo desideri)

\address{via San Cilino~113, Trieste}{CAP~34128}     % indirizzo (opzionale)

\mobile{+39~347~9540982}                     % cellulare (opzionale)

% \phone{+39~0547~123456}                      % telefono fisso (opzionale)

%\fax{+39~0547~123456}                       % fax (opzionale)

\email{adr.fantini@gmail.com ; afantini@ictp.it} % indirizzo e-mail

\extrainfo{ORCID: https://orcid.org/0000-0002-8090-213X}

% \homepage{www.lorenzopantieri.net}           % sito Web (opzionale)

%\extrainfo{Informazioni extra}              % informazioni extra (opzionale)

\photo[64pt][0.4pt]{AF}                      % inserisce una fotografia (opzionale);
                                             % '64pt' è l'altezza della foto;
                                             % '0.4pt' è lo spessore della cornice (si metta '0' per non averla);
                                             % 'LP' è il nome dell'immagine

%\quote{La citazione è un utile sostituto
%       dell'arguzia. \\  --- Oscar Wilde}    % una citazione (opzionale)



\begin{document}

\makecvtitle



\section{Dati anagrafici}

% TODO :
% * separa software e linguaggi di programmazione.
% * aggiungi un po' di cose personali, vedi l'altro CV, quello del PhD
% * aggiungi lavoro con Srdan?
% * aggiungi i temi di cui ti piace di più occuparti (mappe, analisi spaziali)
% * considera di togliere click-edit, ncedit, chymview...
% * considera di aggiungere l'istruzione superiore, vedi altro CV
% * metti tutto su un sito
% * metti un link alla tesi finale, sia di PhD che magistrale
% * aggiungere link a stars
% * corsi seguiti durante triennale, magistrale e dottorato
% * famiglia?
% * fai versione inglese
% * controlla spelling


\cvdoubleitem{Nascita}{Belluno, 6 novembre 1990}{Nazionalità}{Italiana}
\cvdoubleitem{Residenza}{Trieste, via San Cilino~113}{Stato civile}{Libero}

\section{Titoli di studio}
\cventry{Marzo 2019} {Dottorato di ricerca in scienze della terra e meccanica dei fluidi}{}{Abdus Salam International Centre for Theoretical Physics (ICTP), sezione Earth System Physics e Dipartimento di Matematica e Geoscienze, Università di Trieste}{Trieste}{Titolo: Climate change impact on flood hazard over Italy \\ Relatrice: Prof. Erika Coppola (Abdus Salam International Centre for Theoretical Physics, Trieste)
}
\cventry{Marzo 2015}{Laurea Magistrale in Fisica}{}{Università degli Studi di Trieste}{percorso in Fisica Terrestre e dell'Ambiente}{Titolo tesi: Evaluation of the impact of the high resolution for the Alpine
region in a present-day and future scenario simulation with the
regional climate model RegCM4 \\ Relatrice: Prof. Erika Coppola (Abdus Salam International Centre for Theoretical Physics, Trieste). \\ Votazione~110/110 con lode.}
\cventry{Novembre 2012}{Laurea Triennale in Fisica}{}{Università degli Studi di Trieste}{}{Titolo tesi: Utilizzo del microcontrollore Arduino
per la realizzazione di misure nei laboratori didattici dei corsi
di Fisica \\ Relatore: Prof. Edoardo Milotti (Dipartimento di Fisica, Università di Trieste). \\ Votazione~106/110.}
\cventry{2009}{Maturità scientifica}{}{Liceo Scientifico ``G. Galilei'', Belluno}{}{Votazione~86/100}

% I primi tre argomenti di \cventry sono obbligatori; gli altri possono essere lasciati vuoti;
% la sintassi del comando è:
% \cventry{anno}{Titolo di studio}{Istituzione}{Città}{Altre informazioni}{Riga di descrizione 1 \\ Riga di descrizione 2}
% ovvero
% \cventry{anno--anno}{Tipo di lavoro}{Datore di lavoro}{Città}{Altre informazioni}{Riga di descrizione 1 \\ Riga di descrizione 2}



%\renewcommand{\listitemsymbol}{-~}           % cambia il simbolo per gli item degli elenchi

\section{Esperienze lavorative e di studio}
\cventry{Giugno 2018}{Ninth ICTP Workshop on the Theory and Use of Regional Climate Models}{speaker e assistente di laboratorio}{Abdus Salam International Centre for Theoretical Physics (ICTP)}{Trieste}{}
\cventry{Maggio 2018}{European Geosciences Union General Assembly}{poster}{}{Vienna}{}
\cventry{Novembre 2017--Marzo 2019}{Rappresentante degli studenti di dottorato}{}{Dipartimento di Matematica e Geoscienze, Università di Trieste}{Trieste}{}
\cventry{Luglio 2017}{10th HyMeX workshop}{speaker}{}{Barcellona}{}
\cventry{Giugno 2017}{Fourth Workshop on Water Resources in Developing Countries: Hydroclimate Modeling and Analysis Tools}{speaker e assistente di laboratorio}{Abdus Salam International Centre for Theoretical Physics (ICTP)}{Trieste}{}
\cventry{Aprile 2017}{European Geosciences Union General Assembly}{partecipante}{}{Vienna}{}
\cventry{Maggio 2016}{Eight ICTP Workshop on the Theory and Use of Regional Climate Models}{speaker e assistente di laboratorio}{Abdus Salam International Centre for Theoretical Physics (ICTP)}{Trieste}{}
\cventry{Maggio 2016}{ICRC-CORDEX 2016}{partecipante}{}{Stoccolma}{}
\cventry{Aprile 2016}{European Geosciences Union General Assembly}{speaker}{}{Vienna}{}
\cventry{Novembre 2015}{CLIVAR-ICTP Workshop on Decadal Climate Variability and Predictability}{partecipante}{Abdus Salam International Centre for Theoretical Physics (ICTP)}{Trieste}{}
\cventry{Aprile--Ottobre 2015}{Borsa post-lauream}{}{Abdus Salam International Centre for Theoretical Physics (ICTP), sezione Earth System Physics}{Trieste}{}
\cventry{Maggio 2015}{Third Workshop on Water Resources in Developing Countries: Planning and Management in Face of Hydroclimatological Extremes and Variability}{partecipante}{Abdus Salam International Centre for Theoretical Physics (ICTP)}{Trieste}{}
\cventry{Maggio 2014}{Seventh ICTP Workshop on the Theory and Use of Regional Climate Models}{partecipante}{Abdus Salam International Centre for Theoretical Physics (ICTP)}{Trieste}{}
\cventry{2012--2014}{Traduttore freelance inglese-italiano}{}{\url{http://www.notebookcheck.it/}}{}{}




% Elenco delle pubblicazioni generato automaticamente con BibTeX

\renewcommand{\refname}{Pubblicazioni}       % per avere 'Pubblicazioni' invece di 'Riferimenti bibliografici'

\nocite{*}

\bibliographystyle{plain_ita}                % stile plain in italiano;
                                             % il file plain_ita (in dotazione con CV.zip) va nella cartella di lavoro

\bibliography{Pubblicazioni}                 % 'Pubblicazioni' è il nome del database di BibTeX

\section{Lingue conosciute}
\cvdoubleitem{Italiano}{Madrelingua}{Inglese}{Ottimo}

\section{Conoscenze informatiche}
\subsection{Sistemi operativi}
\cvcomputer{Unix/Linux}{Ottima}{Windows}{Buona}

\subsection{Sistemi HPC}
\cvcomputer{PBS (Argo)}{Ottima}{SLURM (Marconi)}{Ottima}

\subsection{Linguaggi e software}
\cvcomputer{R}{Ottima}{Bash/Fish}{Ottima}
\cvcomputer{CDO}{Ottima}{GDAL}{Buona}
\cvcomputer{\LaTeX}{Buona}{NCO}{Buona}
\cvcomputer{Python}{Discreta}{Arduino}{Discreta}
\cvcomputer{GDAL}{Discreta}{SVN/git}{Discreta}
\cvcomputer{C}{Base}{ROOT}{Base}
\cvcomputer{MATLAB}{Base}{Fortran}{Base}
\cvcomputer{QGIS}{Base}{}{}

\subsection{Modelli computazionali}
\cvcomputer{Modello climatico regionale RegCM4}{Buona}{Modello idrologico CHyM}{Base}

\subsection{Contributi software}
\cventry{R package \href{https://github.com/r-spatial/stars/}{stars}}{spatiotemporal tidy arrays for R}{collaboratore}{}{}{}
\cventry{ncEdit}{software per la modifica visuale di dati NetCDF grigliati}{autore}{versione beta per uso interno}{}{}
\cventry{CHyMView}{software per l'analisi dei dati di output del modello idrologico CHyM}{autore}{versione alpha per uso personale}{}{}

% \section{Allegati}
% \cvitem{1}{Mettere qui allegati?}
% \cvitem{2}{Esami sostenuti nel corso di laurea in Matematica}
% \cvitem{3}{Sommario dell'\emph{Arte di scrivere con \LaTeX}}

\section{Altri interessi}
\cvitem{}{\begin{itemize}
 \item Montagna, alpinismo ed arrampicata
 \item Informatica e tecnologia
 \item Meccanica ed elettronica
\end{itemize}}


% \clearpage
%
% \section{Allegato 1: Sommario della tesi di laurea in Matematica}
% \cvitem{Titolo}{Le origini della teoria delle distribuzioni}
% \cvitem{Relatore}{Prof.~Ermanno Lanconelli}
% \cvitem{Data esame}{17 novembre 2000}
% \subsection{Sommario}
%
% \cvitem{}{Il lavoro di tesi valuta le motivazioni che diedero origine alla teoria delle distribuzioni.}
%
% \cvitem{}{Obiettivo specifico del lavoro è mostrare come le ricerche dei matematici impegnati nel problema della fondazione del concetto di distribuzione non siano avvenute sulla spinta di una sempre maggiore generalizzazione dei risultati o a causa di astratte questioni di rigore: alla formulazione di nuovi standard di rigore si perviene quando i vecchi criteri non permettono una risposta adeguata alle domande che vengono dalla pratica matematica o addirittura da problemi in certo senso esterni alla matematica che, trattati matematicamente, impongono mutamenti del quadro teorico.}
%
% \cvitem{}{In questo modo, è possibile accorgersi della differenza che intercorre fra la matematica come sistema organicamente strutturato e come disciplina \emph{in fieri}: spesso gli obiettivi concettuali raggiunti in un dato momento storico non sono che un punto di partenza, e risulta evidente che molte lacune devono essere colmate o che le estensioni veramente importanti devono ancora essere realizzate.}
%
%
%
% \section{Allegato 2: Esami sostenuti nel Corso di laurea in Matematica}
% \cvitem{}{%
% \begin{tabular}{l@{\quad}l}
% Algebra                           & 30/30 e lode \\
% Calcolo Numerico e Programmazione & 30/30 e lode \\
% Istituzioni di Analisi Superiore  & 30/30 e lode \\
% Geometria II                      & 30/30 e lode \\
% Critica dei Principi              & 30/30 e lode \\
% Calcolo delle Probabilità         & 30/30 e lode \\
% Analisi Superiore                 & 30/30 e lode \\
% Matematiche Elementari            & 30/30 e lode \\
% Geometria Superiore               & 30/30 e lode \\
% \end{tabular}
% }



% \section{Allegato 3: Sommario dell'\emph{Arte di scrivere con \LaTeX}}
% \cvitem{}{\LaTeX{} è un programma di composizione tipografica liberamente disponibile, indicato soprattutto nella preparazione di documenti scientifici ai più elevati livelli di qualità. Lo scopo del lavoro, rivolto sia a chi muove i primi passi in \LaTeX{} sia a quanti già lo conoscono, è di offrire ai suoi utenti italiani le conoscenze essenziali per poterlo usare con successo.}
%
% \cvitem{}{I concetti fondamentali della materia, raccolti da svariati manuali, vengono presentati nel modo più chiaro e organico possibile; nel contempo si fornisce un vasto campionario di esempi e si analizzano alcuni problemi tipici della stesura di una pubblicazione scientifica o professionale in italiano, indicando per ciascuno le soluzioni per noi migliori.}
%
%
%
% % Lettera di presentazione
%
% \clearpage
%
% \recipient{Prof.~Enrico Gregorio}%           % dati del destinatario
%           {Università di Verona \\
%            viale dei Giardini, 15 \\
%            37134 Verona}
%
% %\date{21 settembre 2012}                    % data
%
% \opening{Egregio prof.~Gregorio,}            % apertura
%
% \closing{In fede,}                           % chiusura
%
% \renewcommand{\enclname}{Allegati}           % per avere 'Allegati' invece di 'Enclosure'
%
% \enclosure{curriculum vitae}                 % allegati
%
% \makelettertitle
%
% \lipsum[1-2]
%
% Albert Einstein ha scoperto che $E=mc^2$ nel 1905.
%
% \[
% e^{i\pi}+1=0.
% \]
%
% \makeletterclosing

\end{document}
